\documentclass{article}
\author{Kaloyan Penev}
\title{Fitting for the PSF in an image}
\begin{document}
\maketitle
\section{Selecting Image Pixels for the Fit}

The goal is to select a set of pixels the values of which will be used to fit
for the PSF in an image. We would like to select a set of pixels belonging to
isolated sources and which contain significant amount of light from the
source they are assigned to.

Some definitions:
\begin{itemize}
	\item[$m_p$] The measured electron counts of pixel $p$
	\item[$(x_n,y_n)$] The position of the $n$-th source on the image
	\item[$B_n, b_n$] The background and its standard variance as estimated
		for the $n$-th source.
	\item[$\alpha$] A tunable parameter setting how far above the standard
		deviation a pixel value should be in order for the pixel to be
		included in the fit for a source PSF.
	\item[$P$] A tunable parameter selecting how many pixel at minimum a
		source must include in order to be included in the PSF fit.
\end{itemize}

We will assume that we start with a list of source positions $(x_n, y_n)$,
backgrounds $(B_n)$ and background variances $(b_n)$. Fit pixels are then
selected following these steps for each source:

\begin{enumerate}
	\item Starting from the pixel which contains the source center, move in
		the positive and negative y direction until the condition
		$m_p^2>B_n+\alpha\left(m_p+b_n\right)$ is no longer satisfied, adding all
		pixels to the list of pixels belonging to source $n$.
	\item For each of the previously selected pixels perform the same
		procedure, but moving in the positive and negative x directions.
	\item If a pixel is included in the pixel lists for more than one source
		eliminate all sources containing that pixel from the fit.
	\item Eliminate bad pixels
	\item Mark saturated pixels
	\item Eliminate sources for which the number of pixels is less than $P$
\end{enumerate}

\section{Fitting a Fixed PSF Model}

We assume that some model containing some (small) number of free parameters
is available for individual source background subtracted PSFs which only
depends on the position of the source on the image, and that changing the
brightness of a source only changes the overall scaling of the PSF.

Given some values for the model parameters, the quantity to minimize ($R$) is
calculated using the following steps:

\begin{enumerate}
	\item For each source ($n$), using local polynomial approximation to each
		PSF we calculate the integral over each selected pixel ($f_{ni}$) to
		a fractional precision of $0.01(m_p+b_n)/(m_p-B_n)$.
	\item The amplitude for the model PSF of each source is estimated as:
		\begin{displaymath}
			A_n\equiv\frac{\sum_p \frac{(m_p-B_n) f_{np}}{m_p+b_n}}{
			\sum\frac{f_{np}^2}{m_p+b_n}}
		\end{displaymath}
	\item For each saturated pixel where $A_nf_{np}$ is more than $m_p$, replace
		$m_p$ with $A_nf_{np}$.
	\item 
		\begin{displaymath}
			R\equiv\sum_n \sum_p\frac{\left(m_p-B_n-A_nf_{np}\right)^2}{m_p+b_n}
		\end{displaymath}
\end{enumerate}

Two modes of PSF fitting are supported:
%
\begin{itemize}
%
	\item[Source by source:] the PSF parameters for each source are derived
		independently
%
	\item[Ensemble:] the PSF parameters are assumed to vary smoothly accross
		the image. A low order polynomial is used to represent each parameter
		and the fit is done on the coefficients of all polynomials such that
		the combined residuals from all source pixels are minimized
		simultaneously. Fitting is initially done with constant PSF
		parameters, once the best fit for those is found, linear terms are
		added and so on, until the prescribed polynomial order is reached.
\end{itemize}

The minimization is done by the Newton-Raphson method, using analytic
first and second derivatives of the residual with respect to the source
parameters in source by source mode or the polynomial expansion
coefficients is ensemble mode. This means that a nearby local minimum of
the residuals is found, so a reasonable initial guess for the PSF
parameters is of some importance.

\section{Calculating Flux}
In order to calculate the flux from amplitude and PSF parameters:
\begin{equation}
	\int_{-\infty}^\infty \int_{-\infty}^\infty
	\exp\left\{-\frac{1}{2}\left[S(x^2+y^2)+D(x^2-y^2)+2Kxy\right]\right\}
	dx dy = \frac{2\pi A}{\sqrt{S^2-D^2-K^2}}
\end{equation}
Under the conditinos that $S^2>D^2+K^2$ and $S+D>0$.
\end{document}
